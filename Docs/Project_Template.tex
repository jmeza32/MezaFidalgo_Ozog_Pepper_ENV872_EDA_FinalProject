% Options for packages loaded elsewhere
\PassOptionsToPackage{unicode}{hyperref}
\PassOptionsToPackage{hyphens}{url}
%
\documentclass[
  12pt,
]{article}
\title{Marine Mammal and Sea Turtle Stranding Analysis}
\usepackage{etoolbox}
\makeatletter
\providecommand{\subtitle}[1]{% add subtitle to \maketitle
  \apptocmd{\@title}{\par {\large #1 \par}}{}{}
}
\makeatother
\subtitle{\url{https://github.com/jmeza32/MezaFidalgo_Ozog_Pepper_ENV872_EDA_FinalProject.git}}
\author{Joshua Meza-Fidalgo, Jess Ozog \& Britney Pepper}
\date{}

\usepackage{amsmath,amssymb}
\usepackage{lmodern}
\usepackage{iftex}
\ifPDFTeX
  \usepackage[T1]{fontenc}
  \usepackage[utf8]{inputenc}
  \usepackage{textcomp} % provide euro and other symbols
\else % if luatex or xetex
  \usepackage{unicode-math}
  \defaultfontfeatures{Scale=MatchLowercase}
  \defaultfontfeatures[\rmfamily]{Ligatures=TeX,Scale=1}
  \setmainfont[]{Times New Roman}
\fi
% Use upquote if available, for straight quotes in verbatim environments
\IfFileExists{upquote.sty}{\usepackage{upquote}}{}
\IfFileExists{microtype.sty}{% use microtype if available
  \usepackage[]{microtype}
  \UseMicrotypeSet[protrusion]{basicmath} % disable protrusion for tt fonts
}{}
\makeatletter
\@ifundefined{KOMAClassName}{% if non-KOMA class
  \IfFileExists{parskip.sty}{%
    \usepackage{parskip}
  }{% else
    \setlength{\parindent}{0pt}
    \setlength{\parskip}{6pt plus 2pt minus 1pt}}
}{% if KOMA class
  \KOMAoptions{parskip=half}}
\makeatother
\usepackage{xcolor}
\IfFileExists{xurl.sty}{\usepackage{xurl}}{} % add URL line breaks if available
\IfFileExists{bookmark.sty}{\usepackage{bookmark}}{\usepackage{hyperref}}
\hypersetup{
  pdftitle={Marine Mammal and Sea Turtle Stranding Analysis},
  pdfauthor={Joshua Meza-Fidalgo, Jess Ozog \& Britney Pepper},
  hidelinks,
  pdfcreator={LaTeX via pandoc}}
\urlstyle{same} % disable monospaced font for URLs
\usepackage[margin=2.54cm]{geometry}
\usepackage{graphicx}
\makeatletter
\def\maxwidth{\ifdim\Gin@nat@width>\linewidth\linewidth\else\Gin@nat@width\fi}
\def\maxheight{\ifdim\Gin@nat@height>\textheight\textheight\else\Gin@nat@height\fi}
\makeatother
% Scale images if necessary, so that they will not overflow the page
% margins by default, and it is still possible to overwrite the defaults
% using explicit options in \includegraphics[width, height, ...]{}
\setkeys{Gin}{width=\maxwidth,height=\maxheight,keepaspectratio}
% Set default figure placement to htbp
\makeatletter
\def\fps@figure{htbp}
\makeatother
\setlength{\emergencystretch}{3em} % prevent overfull lines
\providecommand{\tightlist}{%
  \setlength{\itemsep}{0pt}\setlength{\parskip}{0pt}}
\setcounter{secnumdepth}{5}
\ifLuaTeX
  \usepackage{selnolig}  % disable illegal ligatures
\fi

\begin{document}
\maketitle

\begin{center}\rule{0.5\linewidth}{0.5pt}\end{center}

\newpage
\tableofcontents 
\newpage
\listoftables 
\newpage
\listoffigures 
\newpage

\hypertarget{rationale-and-research-questions}{%
\section{Rationale and Research
Questions}\label{rationale-and-research-questions}}

\begin{itemize}
\tightlist
\item
  Contains clear context for research topic
\item
  Contains rationale for dataset of choice
\item
  Contains one or more questions of an appropriate scope for the project
\end{itemize}

Aquarium scientists maintain data on marine mammal and sea turtle
strandings and sightings as a means of monitoring the behavior and
health of these animals, always vigilant for changes in patterns that
might signal an unusual event such as a viral outbreak or a toxic algal
bloom. The condition of these animals reveals much about the health of
our oceans.

Marine animal strandings can be an indicator of the health of our
oceans. Seeing patterns in the strandings of marine mammals and sea
turtles can be indicative of viral outbreaks or toxic algal blooms.
Being able to monitor these animals -heath impacts on humans
--\textgreater{} it something is harming animals, will it harm us when
we consume seafood -monitor over the years to see if there were any
trends on increases in strandings

\newpage

\hypertarget{dataset-information}{%
\section{Dataset Information}\label{dataset-information}}

\begin{itemize}
\tightlist
\item
  Describes source and content of data
\item
  Details the wrangling process from raw to processed data
\item
  Contains a table summarizing the dataset structure
\end{itemize}

Mystic Aquarium's marine mammal and sea turtle stranding data 1976-2011
-contains whales, dolphins, and sea turtles -number of different species
in each family -datum WGS:1984 -transforming to UTM --\textgreater{} 18
(32619) or 19 () -coordinate system: -raw data -output table
--\textgreater{} take from data\_wrangling file

\newpage

\hypertarget{exploratory-analysis}{%
\section{Exploratory Analysis}\label{exploratory-analysis}}

\begin{itemize}
\tightlist
\item
  Flow between text and visualizations is cohesive
\item
  Relevant exploratory information is visualized
\end{itemize}

\newpage

\hypertarget{analysis}{%
\section{Analysis}\label{analysis}}

\begin{itemize}
\tightlist
\item
  Flow between text and visualizations is cohesive
\item
  Visualizations and statistical tests pertain directly to specific
  questions
\item
  Statistical results are reported in plain language with relevant
  statistical output in parentheses
\item
  Findings are reported clearly in relation to research questions
\end{itemize}

\hypertarget{pinnipeds}{%
\subsection{\texorpdfstring{\textbf{Pinnipeds}:}{Pinnipeds:}}\label{pinnipeds}}

\emph{Question 1}: \textless insert specific question here and add
additional subsections for additional questions below, if
needed\textgreater{}

\hypertarget{whales}{%
\subsection{\texorpdfstring{\textbf{Whales}:}{Whales:}}\label{whales}}

\emph{Question 1}: \textless insert specific question here and add
additional subsections for additional questions below, if
needed\textgreater{}

\hypertarget{turtles}{%
\subsection{\texorpdfstring{\textbf{Turtles}:}{Turtles:}}\label{turtles}}

\emph{Question 1}: \textless insert specific question here and add
additional subsections for additional questions below, if
needed\textgreater{}

\newpage

\hypertarget{summary-and-conclusions}{%
\section{Summary and Conclusions}\label{summary-and-conclusions}}

\begin{itemize}
\tightlist
\item
  Major findings are summarized
\item
  Conclusions relate back to the original research context
\end{itemize}

\newpage

\hypertarget{references}{%
\section{References}\label{references}}

\textbf{Data}: \url{https://seamap.env.duke.edu/dataset/945}

\begin{enumerate}
\def\labelenumi{\arabic{enumi}.}
\item
  Halpin, P.N., A.J. Read, E. Fujioka, B.D. Best, B. Donnelly, L.J.
  Hazen, C. Kot, K. Urian, E. LaBrecque, A. Dimatteo, J. Cleary, C.
  Good, L.B. Crowder, and K.D. Hyrenbach. 2009. OBIS-SEAMAP: The world
  data center for marine mammal, sea bird, and sea turtle distributions.
  Oceanography. 22(2):104-115.
\item
  Smith, A. 2014. Mystic Aquarium's marine mammal and sea turtle
  stranding data 1976-2011. Data downloaded from OBIS-SEAMAP
  (\url{http://seamap.env.duke.edu/dataset/945}) on 2022-04-02.
\end{enumerate}

\end{document}
